Based on the four themes identified in the panel, we propose the following conclusions:
    
\textbf{Validation of the Web-First Approach}: The increasing presence of tools built with WebXR confirms that open web standards and live-reprogrammable software stacks effectively lower the "tech barrier" for immersive analytics, both on the developer and the user side.

\textbf{The Demise of the "Horse Race"}: The goal is not to replace 2D with 3D, or to only fill 3D into “gaps” where 2D does not work, but to achieve functional integration where legacy toolkits (\textit{D3}/\textit{Vega-Lite}) and immersive toolkits coexist.

\textbf{Shift from "One-Off" Applications to Scaffolds}: There is a need to move from building specialized tools (and demonstration projects) to mature "scaffold toolboxes" that allow for live, collaborative modifications in a persistent environment.

\textbf{Learning from Physicalization}: Findings from data physicalization research show that material constraints, bodily manipulation, and spatial arrangement strongly influence sense-making and memory \cite{pahr_nodkant_2025, pahr_squishicalization_2025}. These insights can inform immersive analytics in XR by providing new design parameters as XR enables the usage of the user’s body as a somewhat tactile input device.

\textbf{XR as a Continuum of Visualization}: Immersive analytics should be understood as part of a continuum from screen-based to tangible and immersive visualization. Frameworks that span these modalities can better support embodied cognition, theory-grounded design, and practical adoption.

\textbf{AI-Assisted Specification}: Some proof-of-concept LLM integrations exist. These could be expanded to help non-experts translate natural language or voice commands into 3D visualization specifications.

\textbf{Long-Term Empirical Studies}: Transitioning from one-off system demonstration to controlled longitudinal user studies with domain experts could enable us to measure actual efficiency gains in daily workflows.